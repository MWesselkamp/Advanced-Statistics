\documentclass{beamer}
\usepackage{amsmath, amssymb}
\usepackage{graphicx}
\usepackage{listings}
\usepackage{color}

% Set up R code formatting
\definecolor{codegreen}{rgb}{0,0.6,0}
\definecolor{codegray}{rgb}{0.5,0.5,0.5}
\definecolor{codepurple}{rgb}{0.58,0,0.82}
\definecolor{backcolour}{rgb}{0.95,0.95,0.92}
\lstdefinestyle{Rstyle}{
    backgroundcolor=\color{backcolour},
    commentstyle=\color{codegreen},
    keywordstyle=\color{magenta},
    numberstyle=\tiny\color{codegray},
    stringstyle=\color{codepurple},
    basicstyle=\ttfamily\footnotesize,
    breakatwhitespace=false,
    breaklines=true,
    captionpos=b,
    keepspaces=true,
    numbers=left,
    numbersep=5pt,
    showspaces=false,
    showstringspaces=false,
    showtabs=false,
    tabsize=2
}

\title{Mixed Effects Models - Day 6}
\subtitle{Praxis and Fitting of Mixed Effects Models in R}
\author{Marieke Wesselkamp\\Department of Biometry and Environmental Systems Analysis\\Albert-Ludwigs-University of Freiburg (Germany)}
\date{February 2023}

\begin{document}

\frame{\titlepage}

\begin{frame}
    \frametitle{The Linear Mixed Effects Model}
    \[
    \mathbf{y} = \mathbf{X} \cdot \mathbf{b} + \mathbf{Z} \cdot \mathbf{u} + \mathbf{e}
    \]
    \[
    \mathbf{e} \sim \mathcal{N}(0, \mathbf{R}), \quad \mathbf{u} \sim \mathcal{N}(0, \mathbf{G}), \quad \mathbf{u} \bot \mathbf{e}
    \]
    where:
    \begin{itemize}
        \item $\mathbf{y}$: measured response values
        \item $\mathbf{X}$: Fixed Effects design matrix
        \item $\mathbf{b}$: Fixed Effects parameter vector
        \item $\mathbf{e}$: errors $\epsilon \sim \mathcal{N}(0, \mathbf{R})$
    \end{itemize}
\end{frame}

\begin{frame}
    \frametitle{Stochastic Components of Mixed Effects Models}
    \begin{itemize}
        \item The first stochastic part describes how random effects parameters $\mathbf{u}$ vary around 0: 
        \[
        \mathbf{u} \sim \mathcal{N}(0, \mathbf{G})
        \]
        \item The second stochastic part describes the residual variance $\mathbf{e}$:
        \[
        \mathbf{e} \sim \mathcal{N}(0, \mathbf{R})
        \]
    \end{itemize}
\end{frame}

\begin{frame}[fragile]
    \frametitle{Orthodont Data Example}
    \textbf{The size (distance) of 16 boys and 11 girls measured 4 times at ages 8, 10, 12, 14}
    \lstset{style=Rstyle}
    \begin{lstlisting}
    library(nlme)
    library(lme4)
    library(lattice)
    data("Orthodont")
    attach(Orthodont)

    str(Orthodont, give.attr = FALSE, vec.length = 2)
    xtabs(~Sex+age, Orthodont)
    \end{lstlisting}
\end{frame}

\begin{frame}[fragile]
    \frametitle{Data Plot: Distance vs Age}
    \lstset{style=Rstyle}
    \begin{lstlisting}
    xyplot(distance ~ age|Sex, groups = Subject, type = "o")
    \end{lstlisting}
    \includegraphics[width=\textwidth]{your_plot.png}
\end{frame}

\begin{frame}[fragile]
    \frametitle{Avoidance of LMM I: Taking Averages}
    \lstset{style=Rstyle}
    \begin{lstlisting}
    dist <- tapply(distance, Subject, mean)
    sex <- as.factor(c(rep("boy", 16), rep("girl", 11)))
    mod <- lm(dist ~ sex)
    summary(mod)$coef
    \end{lstlisting}
    \textbf{Loss of information: no age-effect anymore!}
\end{frame}

\begin{frame}[fragile]
    \frametitle{Fitting a Linear Mixed Effects Model}
    \lstset{style=Rstyle}
    \begin{lstlisting}
    mod.lmer.1 <- lmer(distance ~ I(age-8) * Sex + (I(age-8)|Subject))
    \end{lstlisting}
    Subtracting the minimum age of 8 years from the age-variable allows for more meaningful interpretation.
\end{frame}

\begin{frame}[fragile]
    \frametitle{Extracting Variance Components}
    From the G-side matrix:
    \lstset{style=Rstyle}
    \begin{lstlisting}
    VarCorr(mod.lmer.1)
    1.7983231 ^ 2 # Random intercept variance at age 8
    \end{lstlisting}
\end{frame}

\begin{frame}[fragile]
    \frametitle{Syntax for `gls` Models}
    \textbf{Fitting a marginal model using `gls` with AR1 correlation structure:}
    \lstset{style=Rstyle}
    \begin{lstlisting}
    mod.gls <- gls(distance ~ I(age-8) * Sex, correlation = corAR1(form = ~ 1|Subject))
    summary(mod.gls)
    \end{lstlisting}
\end{frame}

\begin{frame}[fragile]
    \frametitle{Extracting the AR1 Variance-Covariance Matrix}
    \lstset{style=Rstyle}
    \begin{lstlisting}
    getVarCov(mod.gls)
    \end{lstlisting}
\end{frame}

\begin{frame}
    \frametitle{Recipe for Building a Mixed Effects Model}
    \begin{itemize}
        \item Define the fixed effects and their interactions
        \item Choose the distribution for the errors
        \item Identify the grouping and random effects
        \item Check for nesting or crossing of random effects
        \item Specify random slopes or contrasts
    \end{itemize}
\end{frame}

\begin{frame}
    \frametitle{Spot the Grouping in the Data}
    \begin{itemize}
        \item How many random effects are there in the data?
        \item If two or more, are they nested or crossed?
        \item Can you specify random slopes or random contrasts for one or all of the random effects?
    \end{itemize}
\end{frame}

\begin{frame}[fragile]
    \frametitle{R Code: Data Plot}
    \lstset{style=Rstyle}
    \begin{lstlisting}
    xyplot(distance ~ age | Sex, groups = Subject, type = "o")
    \end{lstlisting}
\end{frame}

\begin{frame}[fragile]
    \frametitle{Fitting a Linear Mixed Effects Model}
    \lstset{style=Rstyle}
    \begin{lstlisting}
    mod.lmer.1 <- lmer(distance ~ I(age-8) * Sex + (I(age-8)|Subject))
    \end{lstlisting}
    Subtraction of the minimum age (8 years) from the age variable makes interpretation more meaningful.
\end{frame}

\begin{frame}[fragile]
    \frametitle{Extracting Variance Components}
    \lstset{style=Rstyle}
    \begin{lstlisting}
    VarCorr(mod.lmer.1)
    1.7983231 ^ 2
    \end{lstlisting}
\end{frame}

\begin{frame}[fragile]
    \frametitle{Variance-Covariance Matrix of Parameters}
    \lstset{style=Rstyle}
    \begin{lstlisting}
    round(vcov(mod.lmer.1), 2)
    \end{lstlisting}
\end{frame}

\begin{frame}[fragile]
    \frametitle{Combined Variance-Covariance Matrix of Residuals}
    \lstset{style=Rstyle}
    \begin{lstlisting}
    var.d <- crossprod(getME(mod.lmer.1,"Lambdat"))
    Zt <- getME(mod.lmer.1,"Zt")
    vres <- sigma(mod.lmer.1)^2
    var.b <- (t(Zt) %*% var.d %*% Zt)
    sI <- vres * Diagonal(nrow(Orthodont))
    var.y <- var.b + sI
    image(var.y)
    \end{lstlisting}
\end{frame}

\begin{frame}
    \frametitle{Keep It Maximal Principle}
    \small{
    "We show that LMMs generalize best when they include the maximal random effects structure justified by the design." \\
    \textit{Barr et al. (2013)}
    }
\end{frame}

\begin{frame}[fragile]
    \frametitle{Random Intercept - Random Slope}
    \lstset{style=Rstyle}
    \begin{lstlisting}
    mod.lmer.1 <- lmer(distance ~ I(age-8) * Sex + (I(age-8) | Subject))
    VarCorr(mod.lmer.1)
    \end{lstlisting}
\end{frame}

\begin{frame}[fragile]
    \frametitle{Random Intercept Without Correlation}
    \lstset{style=Rstyle}
    \begin{lstlisting}
    mod.lmer.2 <- lmer(distance ~ I(age-8) * Sex + (I(age-8) || Subject))
    VarCorr(mod.lmer.2)
    \end{lstlisting}
\end{frame}

\begin{frame}[fragile]
    \frametitle{Random Intercept Only}
    \lstset{style=Rstyle}
    \begin{lstlisting}
    mod.lmer.3 <- lmer(distance ~ I(age-8) * Sex + (1 | Subject))
    VarCorr(mod.lmer.3)
    \end{lstlisting}
\end{frame}

\begin{frame}[fragile]
    \frametitle{Design Matrix for Predictions}
    \lstset{style=Rstyle}
    \begin{lstlisting}
    newdata.lmer <- data.frame(
      age = rep(seq(8, 14, 0.1), 27),
      Sex = c(rep("Male", 16 * 61), rep("Female", 11 * 61)),
      Subject = rep(levels(Orthodont$Subject), each = 61),
      distance = rep(NA, 27 * 61))
    \end{lstlisting}
\end{frame}

\begin{frame}[fragile]
    \frametitle{Population-Average Predictions}
    \lstset{style=Rstyle}
    \begin{lstlisting}
    newdata.lmer$distance <- predict(mod.lmer.1, newdata.lmer, re.form = NA)
    xyplot(distance ~ age | Subject, data = newdata.lmer, type = "l", lwd = 3)
    \end{lstlisting}
\end{frame}

\begin{frame}[fragile]
    \frametitle{Confidence Intervals for Population Trend}
    \lstset{style=Rstyle}
    \begin{lstlisting}
    X <- model.matrix(terms(mod.lmer.1), newdata.lmer)
    XVX <- X %*% vcov(mod.lmer.1) %*% t(X)
    se <- sqrt(diag(XVX))
    conf <- se * 1.96
    newdata.lmer <- data.frame(newdata.lmer, lo = newdata.lmer$distance - conf, hi = newdata.lmer$distance + conf)
    \end{lstlisting}
\end{frame}

\begin{frame}[fragile]
    \frametitle{Plot Confidence Intervals}
    \lstset{style=Rstyle}
    \begin{lstlisting}
    plot(distance ~ age, Orthodont[1:64,], type = "p", col = "blue")
    lines(newdata.lmer$age[newdata.lmer$Sex =="Male"], newdata.lmer$distance[newdata.lmer$Sex=="Male"], lwd = 4, col = "darkblue")
    lines(newdata.lmer$age[newdata.lmer$Sex =="Female"], newdata.lmer$distance[newdata.lmer$Sex=="Female"], lwd = 4, col = "darkred")
    \end{lstlisting}
\end{frame}

\begin{frame}
    \frametitle{Recap of Day 6}
    \begin{itemize}
        \item Fitting Linear Mixed Effects Models using `lme4` and `nlme`
        \item Extracting Random Effects parameters using `VarCorr`
        \item Population-average and BLUP predictions
        \item Confidence intervals for fixed and random effects
    \end{itemize}
    Afternoon exercises on fitting mixed effects models in R.
\end{frame}

\end{document}