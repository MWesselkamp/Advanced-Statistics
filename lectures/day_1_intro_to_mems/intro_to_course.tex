\documentclass{beamer}
\usepackage{graphicx}
\usepackage{hyperref}
\usepackage{amsmath}
\usepackage{ulem}

\title{Mixed Effects Models - Day 1}
\subtitle{Introduction to Course}
\author{Marieke Wesselkamp}
\institute{Department of Biometry and Environmental Systems Analysis \\
Albert-Ludwigs-University of Freiburg (Germany)}
\date{\today}

\begin{document}

\frame{\titlepage}

\begin{frame}
\frametitle{Course Aims}
\begin{itemize}
    \item Refreshing your statistic basics
    \item Making you familiar with the philosophy and practice of Linear Mixed Effects Models (LMM) and Generalised Mixed Effects Models (GLMM)
    \item Giving you the basics for building, validating, analysing, and interpreting your own LMMs and GLMMs in R
    \item Excite you about the Bayesian approach
    \item Preparing the ground that will allow you to go further from here: there is \textbf{much} more!
\end{itemize}
\end{frame}

\begin{frame}
\frametitle{Time Table}
\begin{itemize}
    \item Wednesday, 14th of October 2024 to Wednesday, 5th of February 2025
    \item From 08:15 - ca. 12:00  (see Syllabus)
    \item From 08:15 - 10:00 Presence-only Lectures and demonstrations, \textbf{unless noted otherwise}
    \item From 10:15 - 12:00 Exercises and discussions + tutorials 
    \item \textbf{Written exam in presence on the 5th of February 2025}
\end{itemize}
\end{frame}

\begin{frame}
\frametitle{Course Content and Schedule}
\begin{itemize}
    \item \textbf{Week 1}: What are Mixed Effects Models, why and when should you use them, and what can you do with them
    \item \textbf{Week 2}: Refreshing Linear Models I
    \item \textbf{Week 3}: Refreshing Linear Models II
    \item \textbf{Week 4}: Generalised Least Squares Models (GLS)
    \item \textbf{Week 5}: Theory of Mixed Effects Models
\end{itemize}
\end{frame}

\begin{frame}
\frametitle{Course Content and Schedule - Week 2}
\begin{itemize}
    \item \textbf{Week 6}: Practice and Fitting of Mixed Effects Models
    \item \textbf{Week 7}: Diagnostics of Mixed Effects Models
    \item \textbf{Week 8}: Inference in Mixed Effects Models
    \item \textbf{Week 9}: Refreshing GLMs
    \item \textbf{Week 10}: Mid-term report
\end{itemize}
\end{frame}

\begin{frame}
\frametitle{Course Content and Schedule - Week 3}
\begin{itemize}
    \item \textbf{Week 11}: Generalised Linear Mixed Models (GLMM)
    \item \textbf{Week 12}: Introduction to Bayesian Statistics
    \item \textbf{Week 13}: Bayesian Mixed Effects Models
    \item \textbf{Week 14}: Recapitulation / question time / old exam
    \item \textbf{Week 15}: Written exam in presence 
\end{itemize}
\end{frame}

\begin{frame}
\frametitle{Course material and lectures}

Join the Github Classroom \href{https://github.com/orgs/EMDS-Statistics/repositories}{here}!

\vspace{2em}

In the tutorate after this lecture:
\begin{itemize}
    \item Create a free Github account.
    \item Become a member of the classroom and get a repository. (with Tutors)
    \item Clone your repository to an R-Project.
    \item Push your exercise for submisson.
\end{itemize}
\vspace{2em}

\end{frame}

\begin{frame}
\frametitle{Exercises}


\begin{itemize}
    \item Clone Materials and pull new content after lecture.
    \item Use the supervised tutorials to solve your exercises.
    \item Submit \textbf{latest Tuesday at 18:00}, before the next lecture!
\end{itemize}
\vspace{2em}

\end{frame}

\begin{frame}
\frametitle{Exam Details}
\begin{itemize}
    \item Wednesday, 5th of February 2025 at 8:00 (s.t.) - 11:30 in CIP Pool 2 of the Herder Building
    \item You will get a data set which you will have to analyse in a meaningful way and draw valid conclusions from this analysis.
    \item The analysis and conclusions have to be submitted as a html file via ILIAS until 12:30 of the exam day.
\end{itemize}
\vspace{2em}
\begin{center}
    \textbf{\color{red}{Register for the exam latest on 1st of February! }}
\end{center}
\end{frame}


\end{document}
