\documentclass{beamer}
\usepackage{amsmath, amssymb}
\usepackage{graphicx}
\usepackage{listings}
\usepackage{color}
\usepackage{subfig}
\usepackage{hyperref}
\usepackage{tcolorbox,fancyvrb,xcolor,tikz}
\tcbuselibrary{skins,breakable}

\usepackage{comment}
% Define a conditional block for skipping
\excludecomment{skipslide}

\newenvironment{VerbatimIN}
 {\VerbatimEnvironment
  \begin{tcolorbox}[
    breakable,
    colback=lightgray,
    spartan
  ]%
  \begin{Verbatim}}
 {\end{Verbatim}\end{tcolorbox}}

 \newenvironment{VerbatimOUT}
 {\VerbatimEnvironment
  \begin{tcolorbox}[
    breakable,
    spartan
  ]%
  \begin{Verbatim}}
 {\end{Verbatim}\end{tcolorbox}}

% R code formatting
\definecolor{codegreen}{rgb}{0,0.6,0}
\definecolor{codegray}{rgb}{0.5,0.5,0.5}
\definecolor{codepurple}{rgb}{0.58,0,0.82}
\definecolor{backcolour}{rgb}{0.95,0.95,0.92}
\lstdefinestyle{Rstyle}{
    backgroundcolor=\color{backcolour},
    commentstyle=\color{codegreen},
    keywordstyle=\color{magenta},
    numberstyle=\tiny\color{codegray},
    stringstyle=\color{codepurple},
    basicstyle=\ttfamily\footnotesize,
    breakatwhitespace=false,
    breaklines=true,
    captionpos=b,
    keepspaces=true,
    numbers=left,
    numbersep=5pt,
    showspaces=false,
    showstringspaces=false,
    showtabs=false,
    tabsize=2
}

\title{Mixed Effects Models - Week 14}
\subtitle{Q \& A}
\author{Marieke Wesselkamp\\Department of Biometry and Environmental Systems Analysis\\Albert-Ludwigs-University of Freiburg (Germany)}
\date{January 2025}

\begin{document}
\frame{\titlepage}

\begin{frame}
    \frametitle{Individual level intercepts: Why?}

    We can understand the why from a Bayesian Perspective. For example, with a biomial GLM with grand population intercept $\beta_0$.
    
    \large
    \begin{align*}
        &s_i \sim Binomial(n_i, p_i) \\
        &logit(p_i) = \beta_0 + \beta_{TANK[i]} \quad \text{for i in 1, ...tanks}\\
        &\beta_{TANK} \sim Normal(0, 5) \\
        &\beta_0 \sim Normal(0, 10) \\
    \end{align*}
    
\end{frame}

\begin{frame}
    \frametitle{Individual level intercepts: Why?}

    Now the GLMM - the hierarchical model. Here, we estimate the variance component of the intercept.
    
    \large
    \begin{align*}
        &s_i \sim Binomial(n_i, p_i) \\
        &logit(p_i) = \beta_0 + \beta_{TANK[i]} \quad \text{for i in 1, ...tanks}\\
        &\beta_{TANK} \sim Normal(\alpha, \sigma) \\
        &\beta_0 \sim Normal(0, 10) \\
        &\alpha \sim Normal(0, 1) \\
        &\sigma \sim HalfCauchy(0, 1) \\
    \end{align*}

    Here, the Gaussian Prior will shrink the possible estimates! And, we estimate the variance from the individual tanks.
    
\end{frame}


\end{document}